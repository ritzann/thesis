A novel approach of calibrating the phase-shift profilometry (PSP) technique for an artifact-free 3D reconstruction is presented in this study. 
Artifacts caused by the fringes and the optical and digital nonlinearities of the camera-projector pair used in the PSP experiment, which include vignetting effect and inherent gamma, are carried out in the 3D reconstruction; hence, these artifacts need to be removed.  
Input-output curve of the camera-projector pair was obtained and gamma inversion was applied through nonlinear fitting of a 4th-degree polynomial function. 
Vignetting effect which causes wrong depth perception of the object was resolved through background subtraction with bicubic interpolation. 
Lastly, filtering in Fourier domain was applied to remove the unwanted sinusoidal fringe artifacts in the unwrapped phase maps. 
The artifacts were successfully removed in the 3D reconstructions of several test objects with known height/depth and replica of some characters in the Angono petroglyphs.
Depth profile analysis of the Angono petroglyphs replica was done to verify the accuracy of the measured phase-to-depth converted phase maps.
Comparison of the actual measured depths using caliper and the calibrated PSP yielded a 3.92\% error.