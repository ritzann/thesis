\section{Algorithm}

\subsection{Phase Wrapping}


\begin{align}
I_1(x,y)&=I_0(x,y) + I_{mod}(x,y)\cos{(\phi(x,y))}, \nonumber\\
I_2(x,y)&=I_0(x,y) + I_{mod}(x,y)\cos{(\phi(x,y)+\frac{\pi}{2})}, \nonumber\\
I_3(x,y)&=I_0(x,y) + I_{mod}(x,y)\cos{(\phi(x,y)+\pi)},\nonumber\\
I_4(x,y)&=I_0(x,y) + I_{mod}(x,y)\cos{(\phi(x,y)+\frac{3\pi}{2})},
\label{eq:four_fringes}
\end{align}  
where $I_0(x,y)$ is the average background (intensity or) grayscale value\footnote{In image processing, intensity and grayscale value are used interchangeably \cite{Gonzales}.}, $I_{mod}(x,y)$ is the grayscale value of the modulated or distorted pattern, and $I_i$, where $i=1,2,3,4$, is the grayscale value of the $i-$th image. To determine $I_0$, images of the fringe patterns projected onto a reference plane are taken. 
To determine $I_{mod}$, images of the fringe projected object are taken. 
The phase difference, which is related to the grayscale values of the images reflecting back from the distortions of the patterns along the surface height, is given by
\begin{equation}
\phi(x,y)=\tan^{-1}{\left( \frac{I_4(x,y) - I_2(x,y)}{I_1(x,y)- I_3(x,y)}\right)}
\label{eq:phase}
\end{equation}
To show Eq.~(\ref{eq:phase}), expressions in Eq.~(\ref{eq:four_fringes}) are rewritten as 
\begin{align}
I_1(x,y)=I_0(x,y) + I_{mod}(x,y)\cos{(\phi(x,y))}, \nonumber\\
I_2(x,y)=I_0(x,y) - I_{mod}(x,y)\sin{(\phi(x,y))}, \nonumber\\
I_3(x,y)=I_0(x,y) - I_{mod}(x,y)\cos{(\phi(x,y))},\nonumber\\
I_4(x,y)=I_0(x,y) + I_{mod}(x,y)\sin{(\phi(x,y))}.
\label{eq:four_fringes2}
\end{align}  
Subtracting $I_2$ from $I_4$ and $I_3$ from $I_1$ yields
\begin{align}
I_4(x,y)-I_2(x,y)&=2I_{mod}\sin{(\phi(x,y))}, \nonumber\\
I_1(x,y)-I_3(x,y)&=2I_{mod}\cos{(\phi(x,y))}.
\end{align}  
Dividing the first expression by the second, we get
\begin{equation}
\tan(\phi(x,y))=\frac{I_4-I_2}{I_1-I_3}.
\label{eq:tan_phase}
\end{equation}
Getting the inverse of Eq.~(\ref{eq:tan_phase}) results in Eq.~(\ref{eq:phase}).  The discontinuities of the tangent inverse function at $2\pi$ are removed by adding or subtracting multiples of $2\pi$.  This step is known as phase unwrapping. 

\subsection{Phase Unwrapping}
