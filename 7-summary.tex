\chapter{Summary and Conclusions}

Phase-shift profilometry was calibrated successfully by removing the artifacts present in the output phase maps. The source of the optical and digital nonlinearites of the camera and projector were addressed and methods for removing their effect on the phase maps and subsequent 3D reconstruction were proposed. 

The proposed methods for removal of gamma which utilizes the input-output curve and for correcting the vignetting effect which uses the interpolated image itself and/or a use of a reference image are versatile as they are independent of the geometry of the setup or the settings of the camera and calibrations need not be done repeatedly for a particular projector-pair.

Sinusoidal fringe artifacts which remain in the unwrapped phase maps were also removed through filtering in the Fourier domain. Filter masks were designed for each image. The fringe artifacts were significantly reduced as seen on the unwrapped phase maps.

Instead of using the 1D phase unwrapping, quality guided path unwrapping algorithm was implemented for less error propagation and for reduction of singularities.

Phase-to-height calibration was performed and tested for object with known heights and were consequently applied to other objects of nonuniform height including some replica of  the characters in the Angono petroglyphs. 

Our proposed methods proved to be successful with the accurate data obtained from the converted depths/heights in which the depth profile analysis yielded a 3.92\% error.
